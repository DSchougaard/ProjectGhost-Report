\documentclass[a4paper,10pt]{article}
\usepackage[utf8]{inputenc}


\title{	
\textsc{Technical University of Denmark} \\ [25pt] 
\huge Project Agreement
}
\author{Daniel Schougaard \\ \textit{s103446}} 
\date{\normalsize\today}
\begin{document}
\maketitle 


\begin{tabular}{ | l | r | }
	Title (English)		&	Personal Password Manager in the Private Cloud			\\
	Title (Danish)		&	Personlig Password Manager i den Private Sky			\\
	ECTS Points			&	32.5													\\
	Student Number		&	s103446													\\
	Start date 			&	4/1-2016 \textit{(dispensation has been given)}			\\
\end{tabular}


\section{Abstract of Project}
	With the recent LastPass leak\footnote{https://blog.lastpass.com/2015/06/lastpass-security-notice.html/}, more and more people are starting to distrust online services to maintain their passwords. Yet, it is undeniable that storing passwords in a way, that can be accessed from multiple devices is the most convenient. Hence, this project will create a self-hosted -- in the private cloud -- password manager.


	The result of this project, will be a web-based server application, in which users are able to generate, store, and retrieve their passwords, in a very user-friendly way. The user will, as a minimum, be able to access passwords on all of their devices, through a browser.

%\section{Abstract of Project}
%	Over the past few years, privacy on the internet has become a growing issue. As the cloud has become more and more popular for home users to use, so has it for hackers to attack. The recent LastPass security break \footnote{https://blog.lastpass.com/2015/06/lastpass-security-notice.html/} is a shining example of this. One way to avoid this, which I know from personal experience is being used, is to use for instance KeePass, and store the \verb=.kbdx= file in Dropbox. 

%	This, however, is very cumbersome and still relies on the fact, that Dropbox will keep your -- granted encrypted -- password file safe. In this project, a password management system will be designed and implemented. The system will be based around a central server application, running on the users own personal hardware. While the server should allow for several users, the target demographic will be individual users storing their own passwords, as opposed to teams, organisations, and corporations. 


%	The result of this project, will be a web-based server application, in which users are able to generate, store, and retrieve their passwords, in a very user-friendly way. One or more clients will also be developed, allowing the user to access passwords through other means than a browser.
\end{document}

