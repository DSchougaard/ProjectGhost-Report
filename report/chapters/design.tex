\chapter{Designing the System}
	In the following chapter, the overall design of the system is discussed. First and foremost, the basic architechture is decided, based on the requirements. Then more detailed subjects are covered, such as how authentication will work, how the solution keeps the users' data safe, and generally how it will work under the hood - so to speak.

	\section{Architecture}
		As per the requirements specified in section \ref{sec:requirements} on page \pageref{sec:requirements} the solution will have to be distributed. As such, there is two primary basic paradigms which can be used: Peer-to-Peer and Client-Server.

		The Client-Server paradigm might very well be the first thing most people think of, when they hear the word cloud. It provides an easy way to synchronize data between devices, simply because the data delivered from the server always represents the persistent set.


		Over the past few years, peer-to-peer technology has become ever so more appealing to the masses. Applications such as BittorrentSync applies peer-to-peer technology in an effort to synchronise data between devices. Such an approach could easily be adapted to the problem at hand. 




		P2P
		Client-Server

	\section{Target of Deployment}

		-- Which Client?
		Native App
		Web App
		Browser Plugin

	\section{Protocol}
		REST
		SOAP
		Sockets
		Remote Procedure Calls / Remote Procedure Invocation
		JSON-WSP
		Corba

	\section{Authentication}
		\subsection{OAuth1/2}
		\subsection{Basic Authentication}
		\subsection{Tokens}
			\subsubsection{Simple Web Tokens (SWT)}
			\subsubsection{JSON Web Tokens (JWT)}
		
		\subsection{Two-Factor Authentication}

	\section{Securing the API}

	\section{Encryption \& Data Security}
		\subsection{Pseudo Zero Knowledge}


	\section{Storing the Data}
		\section{SQL vs NoSQL}
		\section{SQLite}
	\section{Storage Scheme}

	\section{Naming the Solution}

