\chapter{Conclusion}
	In this thesis, the need for a self-hosted password manager, in the private cloud was argued for and a list of requirements where established. A thorough investigation of available commercial tools and academic solutions were reviewed, and it was found that \emph{none} of the solutions sufficiently fulfilled \emph{all} of the requirements.

	As a response to these findings, a \emph{new} and improved design for a self-hostable password manager is proposed. The design utilizes something dubbed pseudo-zero-knowledge. This denotes that every password is encrypted client side, prior to being sent to the server. This is done to ensure that the users' passwords remain their own, as the server and the server owner has no way of actually decrypting the passwords.

	In accordance with the design specifications, a prototype implementation was developed, and is working to the point where it can be called a minimal viable product. The prototype completely fulfils \emph{every} requirement, specified in section \ref{sec:requirements} on page \pageref{sec:requirements}. As such, it is concluded that the prototype is complete and is fully functional.

	As covered, there \emph{are} risks involved with running this solution, however it is concluded that the benefits far outweigh the risks. Finally, the benchmarking revealed that while the performance is \emph{horrible}, it is \emph{feasible} to host the implementation on a Raspberry Pi 1 Model B+. It is suspected, that using a newer and more powerful model, will in fact improve this. 

	As a \emph{final} remark, it is concluded that the project in its entirety is complete and a success.