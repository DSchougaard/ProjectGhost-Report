\chapter{Analysis}
\label{chap:analysis}
	As with so many projects, toy never start from scratch. In this chapter I will investigate exising solutions solving the same problem -- or one fairly similar to -- as desribed in chapter \ref{chap:intro}.

	\section{Existing Available Tools}
		Having reseached which tools are currently available, I've condensed my findings into the following list. 

		\begin{itemize}
			\item LastPass, and Similar Solutions
			\item KeePass, and Similar Solutions
			\item Rattic
			\item Encryptr
			\item Passwordstate
			\item Vault \emph{(Zoho)}
			\item TeamPasswordManager
			\item Simple Safe
			\item PassWork
			\item SimpleVault
			\item RoboForm
			\item TeamPass (http://teampass.net)
			%\item Vault (https://vaultproject.io/)
		\end{itemize}

		\subsection*{LastPass, and Similar Solutions}
		\label{subsec:lastpass}
			In the name of usability services such as LastPass\footnote{https://lastpass.com/}, PassPack\footnote{https://www.passpack.com/}, DashLane\footnote{https://www.dashlane.com/}, and so many others grew popular, and it is easy to understand why. Enabling you to access your passwords from several devices, through native apps or the browser, it seemed like it was the perfect match of usability and security. To not repeat myself over and over again, I will focus on LastPass \emph{(due to them being the most well-known)} as a representation of this group.

			If we start by looking at the technical details of LastPass, they quote themselves using 256-bit AES encryption, and applies PBKDF2, in order to make it as difficult as possible, to crack stored data. For maximum security, both encryption and decryption, is done client side\cite{lastpass_cleintsideencryption}, as to avoid transferring the actual password, unencrypted, to their servers. Encryption and decryption is done using the master password, which is never actually sent to their servers. Finally, as is to be expected, all connections to LastPass' servers, are SSL encrypted.

			Having examined the technical aspect, we need to pay attention to the usability as well. Looking at their web UI, it shows a reasonably straight forward design. Allowing the user to organise passwords folders, creating a two-level structure, as seen on figure \ref{fig:lastpass_main} on page \pageref{fig:lastpass_main}. While this \emph{does} allow the user some level of organisation, several levels would have been preferable. Additionally, LastPass is renowned for their apps and plugins, covering all major operating systems and browsers, creating a near seamless integration, when it comes to addition of new passwords and auto-filling stored passwords.

			For devices without a browser supporting plugins, LastPass offers a so-called bookmarklet\cite{lastpass_bookmarklet}. A bookmarklet is a bookmark, which essentially contains JavaScript code, in order to add previously unobtainable features, in a browser. While this on the surface seems like a nifty feature, several studies have shown that the bookmarklets represent a significant security threat. In \cite{bookmarklet} they discuss an attack on LastPass, exploiting the users bookmarklet, to gain access to virtually all of the users stored credentials.

			However, with the recent leak from LastPass \cite{lastpass_leak}, more and more users grew suspicious of these services. No matter how much encryption you apply, you can not get around the fact, that you have to \emph{trust} LastPass to both be completely honest about their encryption technology, \emph{and} storing your sensitive data. In many of the more sceptical user's eyes, this is a huge drawback, and why this solution is deemed unusable to solve the problem at hand.


			\begin{figure}[h!]
				\centering
				\includegraphics[width=\textwidth]{figures/analysis/lastpass_main.png}
				\caption{Screenshot of LastPass' main view, in a browser.}
				\label{fig:lastpass_main}
			\end{figure}



		\subsection*{KeePass}
			As the users grew suspicious of LastPass and the likes, a lot of them moved over to e.g. KeePass\footnote{https://keepass.org}, which allows the user to store the passwords in a local file. While there exists a plethora of tools similar to KeePass, I will focus on KeePass as the representative of the bunch, much as in section \ref{subsec:lastpass}.

			If we again start by examining the technical details, as of version 2.x KeePass only -- per default -- offers AES-256 encryption, which is seen on figure \ref{fig:keepass_create_security} on page \pageref{fig:keepass_create_security}, with additional algorithm choices available through plugins \cite{keepass_security}. This enables users to tailor the encryption security, to their own needs -- and beliefs. 

			Looking at the main UI, of which an example is shown on figure \ref{fig:keepass_main} on page \pageref{fig:keepass_main}, KeePass features exactly that which could be improved in LastPass: A tree like structure, in order to completely organise passwords. Other than that, there isn't anything noteworthy to say about their UI: It features the necessary and that's about it. A final thing worth mentioning, is that their password generator is completely customisable, as seen on figure \ref{fig:keepass_newpassword_passwordgen} on page \pageref{fig:keepass_newpassword_passwordgen}. You can manually choose, exactly which character sets, you wish to be in your passwords, enabling you to have passwords using local accents should the target system support it, which is a really nifty feature.


			However, having praised the features of KeePass, it does lack something extremely important: Usability. More precisely, it lacks distribution. Since KeePass works on a local file, it would only inherently work on a \emph{single} device. Should one wish to distribute it, another tool has to be involved. File synchronisation tools, such as Dropbox, Google Drive, or Syncthing could be used, in order to create a distributed-ish feel to KeePass. However, you do still rely on a third party tool, which is a drawback. Additionally, there is the lack of cross-platform compatibility, since \emph{officially} KeePass only supports Windows. Granted, there exists unofficial ports for Linux, OS X, Android, etc., but you have to trust the developers of these unofficial applications. By extension, this introduces the threat of security breaches. Another negative in regards to usability, is the lack of a browser extension. While there exists third party solutions for this, the authors personal experiences with setting these up and using them, is \emph{quite} negative.	

			Hence, all things taken into consideration, while KeePass has its moments it is an less than ideal solution.

			
			\begin{figure}[h!]
				\centering
				\includegraphics[width=\textwidth]{figures/analysis/keepass_mainview.png}
				\caption{Screenshot of KeePass' main view.}
				\label{fig:keepass_main}
			\end{figure}

			\begin{figure}[h!]
				\centering
				\includegraphics[width=0.7\textwidth]{figures/analysis/keepass_create_security.png}
				\caption{Screenshot of KeePass' security options.}
				\label{fig:keepass_create_security}
			\end{figure}
		
			%\begin{figure}[h!]
			%	\centering
			%	\includegraphics[width=\textwidth]{figures/analysis/keepass_newpassword_main.png}
			%	\caption{.}
			%	\label{fig:}
			%\end{figure}

			\begin{figure}[h!]
				\centering
				\includegraphics[width=0.7\textwidth]{figures/analysis/keepass_newpassword_passwordgen.png}
				\caption{Screenshot of KeePass' password generator settings.}
				\label{fig:keepass_newpassword_passwordgen}
			\end{figure}

		\subsection*{Rattic}
			Rattic\cite{rattic_frontpage} represents a third type of password manager, and the primary focus of this thesis: A self-hosted password manager, in the so-called private cloud. While at first glance, Rattic seems to be a solution very suited to the problem described earlier, it becomes a far more sketchy solution, upon investigating it closely. Rattic wonderfully describe their solution as:

			\begin{quote}
				\emph{RatticDB is a password management database designed for humans. We have focused on making it able to manage passwords for a team and to make that as easy as possible.}\\\cite{rattic_frontpage}
			\end{quote}

			Since Rattic \emph{is} meant for teams it has multi-user support. Rattic organises passwords and users in groups, and these groups are used for access control. A group is a collection of users, which can access the same passwords. An example of this could be \verb=DevelopmentTeam1= and \verb=DevelopmentTeam2=. Members of team one, can access their own passwords, and members of team two can access theirs, but unless specifically stated, they can not access each others. Additionally it supports tags for their passwords, allowing for even further organisation, for their users allowing quick access to similar passwords, from across different groups.

			However, the fact that Rattic markets itself at teams, rather than individual users is evident by the fact that as per default, you can not create ``private'' passwords, which only a single user can access. To achieve that, you would need to manually create a new group -- per user -- with only said user as member. While this would work, it is very much a work-around of Rattic's default behaviour, for it to work that way.

			From a user experience point of view, Rattic is a rather friendly tool. On figure \ref{fig:rattic_main} on page \pageref{fig:rattic_main}, you see a clean, minimalistic view of available passwords. Granted, this figure is from an Admin's point of view, hence he can access both the \verb=DevelopmentTeam1= and \verb=DevelopmentTeam2= groups, and consequently passwords stored under them.

			\begin{figure}[h!]
				\centering
				\includegraphics[width=0.95\textwidth]{figures/analysis/rattic_main.png}
				\caption{Rattic's Frontpage of their Web UI.}
				\label{fig:rattic_main}
			\end{figure}

			Adding passwords is just as easy in KeePass, cf. figure \ref{fig:rattic_newpassword_main} on page \pageref{fig:rattic_newpassword_main}. Simply type in the details, select an owner group and submit. While the password generator, cf. figure \ref{fig:rattic_newpassword_passwordgen} on page \pageref{fig:rattic_newpassword_passwordgen}, could stand to have some additional features, it is sufficient for generating strong passwords with high enough entropy. A nifty little feature, is the \verb=Download KeePass= button, which allows a user to download passwords in the KeePass format, making it available for later offline use.

			\begin{figure}[h!]
				\centering
				\includegraphics[width=0.95\textwidth]{figures/analysis/rattic_newpassword_main.png}
				\caption{Adding a new password, in Rattic.}
				\label{fig:rattic_newpassword_main}
			\end{figure}

			\begin{figure}[h!]
				\centering
				\includegraphics[width=0.95\textwidth]{figures/analysis/rattic_newpassword_passwordgen.png}
				\caption{Generating a password in Rattic.}
				\label{fig:rattic_newpassword_passwordgen}
			\end{figure}

			Having said that, there are some technical concerns, regarding Rattic. Rattic does \emph{not} encrypt passwords stored in the database -- something they are very open about. In \cite{rattic_encryption} they argue heavily for their lack of encryption, due to requiring less code and tests. They do, however, highly recommend storing the database on an encrypted drive, to ensure database protection. However, this \emph{does} mean that a sysadmin can access \emph{all} passwords, should he or she have the encryption key for the drive. Due to this, the deployment of Rattic is fairly complex, as they admit themselves. 

			Rattic is developed in Python, using the Django framework and tested on the Apache server.



		\subsection*{Encryptr}
			Bordering between the type of LastPass and Rattic, Enryptr \cite{encryptr} relies on the Crypton\cite{crypton} backend\cite{encryptr_backend}, available hosted at SpiderOak\cite{crypton_spideroak}. Per default, Encryptr \emph{only} supports hosting passwords at the Crypton backend, hosted at SpiderOak. However, it \emph{is} possible to run this in the private cloud, with your own Crypton backend. \emph{But} it requires manually editing source files\cite{encryptr_selfhost}, which makes the setup a pain. So not only would you have to set up Crypton, and its requirements, you would have to download the source of the apps, change the specified line of code, compile and \emph{then} you could use it. Because of this technical aspect, usability is virtually zero, as you would need to be fairly confident behind a keyboard, to successfully set this up.

			Having said that, the UI of Encryptr is \emph{very} minimalistic and sleek, which both is a good and a bad thing. As seen on figure \ref{fig:encryptr_main} on page \pageref{fig:encryptr_main}, all passwords are stored in a \emph{single} list: There is \emph{no} organisation, other than labels. Granted, they do support searching, but somehow that still makes it very confusing, if more than a handful of passwords or secrets are stored, and since Encryptr not only supports storing passwords, but also credit card information and general secrets, this could be achieved fairly quickly.

			Adding a passwords shows something disturbing: Lack of a decent password generator, cf. figure \ref{fig:encryptr_newpassword} on page \pageref{fig:encryptr_newpassword}. Once the form for adding a new password is opened, a password is generated as per some default behaviour. There is no customising entropy of the password, or even something as simple as the length.

			\begin{figure}[h!]
				\centering
				\includegraphics[width=0.75\textwidth]{figures/analysis/encryptr_main.png}
				\caption{Encryptr's main view.}
				\label{fig:encryptr_main}
			\end{figure}


			\begin{figure}[h!]
				\centering
				\includegraphics[width=0.75\textwidth]{figures/analysis/encryptr_newpassword_main.png}
				\caption{Adding a new password in Encryptr.}
				\label{fig:encryptr_newpassword}
			\end{figure}

			On the technical side, Crypton -- the backend -- is actually fairly interesting. Using their own definition of zero-knowledge, the developers created a complete framework for securely transferring and storing data at a remote machine\cite{crypton_paper}. They claim, that it is impossible to obtain the unencrypted data on their servers, without actually getting hold of the users private encryption key. The Crypton backend is open source, and available at \cite{crypton_git}. What powers Crypton, cryptographically speaking, is AES-256.


		\subsection*{Vault}
			Vault\emph{(ZOHO)}\cite{vault_zoho} is another piece of software, that requires you to submit your data to their servers. Where this tool differs from LastPass and Encryptr, is that it is \emph{clearly} aimed at enterprise customers, which is clearly evident based on their features, such as LDAP integration.

			Vault organises the passwords in so called Chambers. Each password \emph{can} be added to one or more chambers, but is not necessary. Seen on figure \ref{fig:vaultzoho_main_secrets} on page \pageref{fig:vaultzoho_main_secrets} we see how it lists all available passwords, in a single list. Using this main list quickly becomes very confusing. Looking at the Chambers list, which is found on figure \ref{fig:vaultzoho_main_chambers} on page \pageref{fig:vaultzoho_main_chambers}, the same pattern arises again. A large list, showing only the important information -- the passwords -- in the smallest part of the UI. All in all, performing simple tasks is very clunky.

			As with Encryptr, password generation lacks a lot. A password can be generated -- and re-generated -- but it happens after some under-the-hood-behaviour, which the user can not choose to customise.


			\begin{figure}[htbp]
				\centering
				\includegraphics[width=0.95\textwidth]{figures/analysis/vaultzoho_main_secrets.png}
				\caption{Listing all stored secrets in Vault \emph{(ZOHO)}.}
				\label{fig:vaultzoho_main_secrets}
			\end{figure}

			\begin{figure}[htbp]
				\centering
				\includegraphics[width=0.95\textwidth]{figures/analysis/vaultzoho_main_chambers.png}
				\caption{Listing Chambers stored in Vault \emph{(ZOHO)}.}
				\label{fig:vaultzoho_main_chambers}
			\end{figure}


			Under the hood, Zoho uses a combination of RSA and AES. To enable sharing, a common AES encryption key is retrieved using RSA keypairs, from the admin, and users involved in sharing\cite{vault_zoho_encryption}.




		\subsection*{TeamPasswordManager}
			TeamPasswordManager\cite{teampasswordmanager_frontpage} \emph{(henceforth referred to as TPM)} is a tool that is fairly similar to Rattic. However, where it differs is the less minimalistic UI. TPM organises passwords in ``Projects'', which is a great indicator that this tool is in fact intended to be used with enterprise in mind. TMP is able to be self-hosted and is platform independent, requiring Apache, PHP and MySQL, which is a great asset.

			Turning to the user experience, the frontpage that the user is presented with, as seen in figure \ref{fig:teampasswordmanager_main} on page \pageref{fig:teampasswordmanager_main}, is a clunky list of all available passwords. For organisational value, groups as ``Favorite'' and ``Recent'' are also available, filtering the available passwords into more manageable lists. But once again, this very clunky and large UI is used, much like in Vault.

			TeamPasswordManager aims itself at -- as the name implies -- teams, much like Rattic. This choice, is very apparent in the work flow. For instance, a password is tied to a ``project", instead of a user. Where TPM excels, is the fact that a project can contain sub-projects, which in return can contain sub-projects, and so forth. This creates a tree-like structure, much like that of KeePass, which can be seen on figure \ref{fig:teampasswordmanager_tree} on page \pageref{fig:teampasswordmanager_tree}.


			\begin{figure}[htbp]
				\centering
				\includegraphics[width=0.95\textwidth]{figures/analysis/teampasswordmanager_main.png}
				\caption{Front page of TeamPasswordManager's web interface.}
				\label{fig:teampasswordmanager_main}
			\end{figure}

			\begin{figure}[htbp]
				\centering
				\includegraphics[width=0.35\textwidth]{figures/analysis/teampasswordmanager_tree.png}
				\caption{The tree-like organisational structure of TeamPasswordManager.}
				\label{fig:teampasswordmanager_tree}
			\end{figure}

			Encryption wise, TPM uses the same basic algorithm we've seen over and over again: AES-256. TPM also uses bcrypt as their chosen key derivation function, to make it as difficult for a potential attacker to brute force password hashes. Using Google's Authenticator, it supports two-factor authentication.


			

			While this software could essentially suffice, in order to meet the requirements, it would be lacking heavily in the user experience department, while also providing features and information, which would only be of use to super-users, and enterprise users.

		\subsection*{Passwordstate}
			Passwordstate from ClickStudios\cite{passwordstate} is a solution some-what similar to both TeamPasswordManager and Vault \emph{(Zoho)}. There is not a seconds doubt, that this is a solution aimed at enterprise users, as they state themselves. This is also evident by the feature set, they have: Sporting not only active directory support, they also have built-in options for High Availability and \emph{several} options for two-factor authentication, amongst others. Passwordstate is self-hostable, but requires a windows platform and the IIS server \cite{passwordstate_requirements}.

			Inspecting the user experience, reveals that Passwordstate is a fresh breeze. This solutions actually manages to give the most important information, the most screen-space. On figure \ref{fig:passwordstate_main} on page \pageref{fig:passwordstate_main}, we see how password state allows for the same tree-like structure, that we've seen before. This creates excellent organisational options, for the users. On the frontpage the user is presented with his or hers favourite passwords list, and recently accessed passwords. Alas, here is where Passwordstate falls short. Instead of keeping a minimalistic UI, the user is presented with information, only interesting to enterprise users, or super-users. A host list, is with a high probability useless for the majority of regular users, as well as the little graph that shows password statistics, as seen on figure \ref{fig:passwordstate_graph} on page \pageref{fig:passwordstate_graph}. This particular graph, presented at this particular place, seems very much to be a graph for the sake of a graph. 

			Additionally, accessing passwords is not straight forward: Again, Passwordstate presents the user with a \emph{lot} of enterprise options, which is far from beneficial for a regular user. The available options can be seen on figure \ref{fig:passwordstate_getpassword} on page \pageref{fig:passwordstate_getpassword}. Taken all of these things into consideration, Passwordstate is an excellent password manager for enterprises, already running Windows as their server software. However, for regular home users, the software is simply \emph{far} too complex.

			Passwordstate has the options to completely customise the rules, that a password is generated by, much like KeePass. On figure \ref{fig:passwordstate_newpassword_passwordgen} on page \pageref{fig:passwordstate_newpassword_passwordgen} you see the \emph{extensive} options, enabling the user to customise the password to their needs.

			\begin{figure}[htbp]
				\centering
				\includegraphics[width=0.95\textwidth]{figures/analysis/passwordstate_main.png}
				\caption{Frontpage of the PasswordState website. \figsrc{fig:passwordstate_main} }
				\label{fig:passwordstate_main}
			\end{figure}

			\begin{figure}[htbp]
				\centering
				\includegraphics[width=0.95\textwidth]{figures/analysis/passwordstate_graph.png}
				\caption{Password statistics graph, in PasswordState. \figsrc{fig:passwordstate_getpassword}}
				\label{fig:passwordstate_graph}
			\end{figure}

			\begin{figure}[htbp]
				\centering
				\includegraphics[width=0.7\textwidth]{figures/analysis/passwordstate_getpassword.png}
				\caption{Retrieving a password in PasswordState. \figsrc{fig:passwordstate_graph}}
				\label{fig:passwordstate_getpassword}
			\end{figure}

			\begin{figure}[htbp]
				\centering
				\includegraphics[width=0.95\textwidth]{figures/analysis/passwordstate_newpassword_passwordgen.png}
				\caption{Generating a new, secure, password in Passwordstate. \figsrc{fig:passwordstate_newpassword_passwordgen}}
				\label{fig:passwordstate_newpassword_passwordgen}
			\end{figure}

			Looking at the technical side, Passwordstate encrypts all passwords using AES-256\cite{passwordstate_security}, which seems to be the standard. Additionally, all sensitive information is salted. Passwordstate protects their sourcecode by the use of:
			\begin{quote}
				\emph{... precompiled ASP.NET pages and obfuscated .NET Assemblies. No longer can web or database administrators gain access to data they are not authorised to view. }\cite{passwordstate_security}
			\end{quote}

			Unfortunately, the limited number of platforms Passwordstate can run on, is a \emph{huge} drawback.

		\subsection*{SimpleSafe}
			SimpleSafe\cite{simplesafe} is another take on the self-hosted team password manager solution, which seems to be the predominant solution available. Unfortunately SimpleSafe's documentation is lacking heavily, but based on their available demo, it appears that all users have access to all passwords. This results in that a single user can not have a private password, for their use only.

			While it appears they have whole-heartedly embraced the idea of a minimalistic design, they've done so at the expense of user experience. The main view is a rather large but accessible list, with customizable attributes. On figure \ref{fig:simplesafe_main} on page \pageref{fig:simplesafe_main} the list is shown. The user can chose to organise passwords \emph{(or custom entries)} into groups. Groups are accessed through a menu, which is auto hidden. The expanded menu is shown on figure \ref{fig:simplesafe_menu} on page \pageref{fig:simplesafe_main}. Unfortunately, changing between groups takes a \emph{long} time. This causes a horrible experience for the user. Generally, the developers of SimpleSafe have been very generous with the animations, as pretty much all actions in the UI invokes some kind of animation. The delays because of this, adds to the overall sluggish feel of the system.

			\begin{figure}[htbp]
				\centering
				\includegraphics[width=0.95\textwidth]{figures/analysis/simplesafe_main.png}
				\caption{Homepage of SimpleSafe}
				\label{fig:simplesafe_main}
			\end{figure}

			\begin{figure}[htbp]
				\centering
				\includegraphics[width=0.70\textwidth]{figures/analysis/simplesafe_groups.png}
				\caption{Menu for changing groups in SimpleSafe.}
				\label{fig:simplesafe_menu}
			\end{figure}

			Security wise, it is \emph{very} difficult to inspect SimpleSafe, due to their own lack of description. The only available information regarding security or encryption is the following quote.
			\begin{quote}
				\emph{SimpleSafe utilises a 256 bit encryption method. Each password has a unique private salt along with a master salt stored separately to the database. Only encrypted passwords are stored within the database.}\cite{simplesafe_faq}
			\end{quote}

			All in all SimpleSafe is a \emph{fairly} sketchy solution. One thing that is pretty neat, is the ability for a user to customise the fields in the different groups. For instance, they allow you to store SSH keys, in a particular field type. This lets the user separate things that require keys / certs, from traditional username/password logins.



		\subsection*{PassWork}
			PassWork\cite{passwork} is yet another take, on the same type of solution as SimpleSafe, Rattic, TeamPasswordManager, and Passwordstate. PassWork is available as both a remote and a self-hosted solution, both of which comes with a price tag. 

			PassWork organises passwords in groups, which in return can contain sub-groups -- or folders as the icon resembles. Each group has a list of users, currently allowed to access passwords in said group. As figure \ref{fig:passwork_adduser} on page \pageref{fig:passwork_adduser} shows, users can be added to groups, with permissions ``Full Access'', ``Edit'', and ``Read''. While differentiating between ``Edit'' and ``Full Access'' might be difficult, the option of setting permissions \emph{that} easily, when adding a user, is surprisingly a very good feature. Per default a group called a ``My passwords'' group is created, in which the user can place private passwords. 

			\begin{figure}[htbp]
				\centering
				\includegraphics[width=0.95\textwidth]{figures/analysis/passwork_adduser_cropped.png}
				\caption{Permissions available, when adding users to a group.}
				\label{fig:passwork_adduser}
			\end{figure}

			PassWork is another example of developers not describing and documenting their soltion properly. Unfortunately it is impossible to completely determine their encryption scheme, but they use ``256-bit passwords'' and RSA keypairs for sharing passwords. 

			Their less than completely transparency when it comes to choice of encryption algorithm not to mention the hefty price-tag, should one wish to self-host it. Additionally, they completely omit to inform which platform(s) the self-hosted version is able to be run on.


		\subsection*{SimpleVault}
			Taking a bit of a different route from other solutions, SimpleVault\cite{simplevault} chooses to encrypt each individual password, with a different ``passphrase''. Based on their user experience, it would seem that SimpleVault is a proof of concept, more than a release software. Something as simple as headers for the table containing the main information is missing, as seen on figure \ref{fig:simplevault_main} on page \pageref{fig:simplevault_main}.

			Adding a password, on figure \ref{fig:simplevault_addpassword}, shows the same lack of attention to the user experience, having two fields completely undescribed. One thing that SimpleVault \emph{does} get right in this context, is the option to generate passwords. Three buttons exists for generating passwords, with increasinly ``rare'' symbols. While options for passwords of specific length lacks, it seems to -- per default -- generate password of suitable length. 

			Retrieving a password, on figure \ref{fig:simplevault_getpassword}, requires the user to type in the passphrase. The choice of this per-password passphrase, will undoubtedly only result in the user using the same password over and over again. However, this choice \emph{does} allow for the system to be used by multiple users, each just using their own master passphrase for all of their passwords.

			\begin{figure}[htbp]
				\centering
				\includegraphics[width=0.95\textwidth]{figures/analysis/simplevault_main.png}
				\caption{Frontpage of SimpleVault.}
				\label{fig:simplevault_main}
			\end{figure}

			\begin{figure}[htbp]
				\centering
				\includegraphics[width=0.95\textwidth]{figures/analysis/simplevault_newpassword.png}
				\caption{Adding a password to be stored in SimpleVault.}
				\label{fig:simplevault_addpassword}
			\end{figure}

			\begin{figure}[htbp]
				\centering
				\begin{subfigure}{\textwidth}
					\centering
					\includegraphics[width=0.95\textwidth]{figures/analysis/simplevault_getpassword.png}
					\caption{Typing passphrase, in order to decrypt stored password.}
					\label{fig:simplevault_getpassword_type}
				\end{subfigure}%
				
				\begin{subfigure}{\textwidth}
					\centering
					\includegraphics[width=0.95\textwidth]{figures/analysis/simplevault_getpassword_decrypted.png}
					\caption{Decrypted password shown in SimpleVault}
					\label{fig:simplevault_getpassword_decrypted}
				\end{subfigure}
				
				\caption{Process of retrieving a password in SimpleVault.}
				\label{fig:simplevault_getpassword}
			\end{figure}


			Security wise SimpleVault is quite horrible. With what can be taken from the quite poor description of the security of the system, the encrypted data is actally \emph{stored in clear text} on the remote machine, at some point or other. When adding a new entry, the password is sent in \emph{clear text} to the server, which then encrypts it. The developer argues that this is fine, as long as only HTTPS is used, for communicating with the server, but regardless of which protocol is used the password is \emph{still} stored in clear text.

		\subsection*{RoboForm}
			RoboForm\cite{roboform} is somewhat a hybrid between the likes of KeePass and LastPass. While at heart, RoboForm is a locally stored password manager, it only exists as a browser extension. Since it only exist as a browser plugin, features such as form auto fill etc. are of course in their repertoire. However, from a ux point of view, the plugin stands out quite a bit from the browser. While this is a relatively small complaint, the usage of design guidelines from both Apple and Google, has helped create platform standards for how apps are supposed look, so the user never feels lost. Additionally, they offer ``RoboForm Everywhere'' cloud synchronisation of the encrypted file. This does however, entail trusting them with safe-keeping the encrypted file.

			Using RoboForm on a single device, renders it as a tool very similar to KeePass, albeit with ``better'' browser integration. Using RoboForm Everywhere results in the same downsides and threats, as previously explained.


		\subsection*{Comparison Charts}


			\begin{tabular} {r l l l }
				Name 				& Web Accessible 	& Bookmarklets	& Self-Hosted 		\\
				\hline
				LastPass 			& \cellcolor{green!75}Yes	& \cellcolor{green!75}Yes	& \cellcolor{red!75} No		\\
				KeePass 			& \cellcolor{red!75}No 		& N/A 						& N/A						\\ 
				Rattic  			& \cellcolor{green!75}Yes 	& \cellcolor{red!75}No 		& \cellcolor{green!75} Yes	\\
				Encryptr 			& \cellcolor{green!75}Yes 	& \cellcolor{red!75}No 		& \cellcolor{green!75} Yes* \\
			\end{tabular}


			\begin{tabular}{ r l l l l }
					Name 				& Desktop Client 	& Web Client 	& Mobile Client 	& Self-Hosted	\\
					\hline
					LastPass 			& Yes 				& Yes			& Yes				& No \\
					KeePass 			& Yes				& No			& \\
					Rattic  			& Yes						& \\
					Encryptr 			& Yes					  	& \\
				\end{tabular}



			\begin{table}
				\begin{tabular}{ r l l l l }
					Name 				& Encryption  & Web Accessible 		& Server Software 		& Language \& Framework 		\\
					\hline
					LastPass 			& AES-256 		& Yes			& N/A 				& N/A \\
					KeePass 			& AES-256 (more available through plugins) & No & N/A & C++ \\
					Rattic 				& None 			& Yes 			& Apache 		& Python \& Django \\
					Encryptr 			& AES-256 		& Yes			& Backend / N/A 	& JS \\
				\end{tabular}
			\end{table}



	\section{Academic Research and Tools}
	\section{Chosen Inspiration}