\chapter{Introduction}
\label{chap:intro}
	For many years, IT professionals have preached the importance of strong passwords. Many publications exists, describing exactly what defines a strong password. The general consensus is, that it needs \emph{at least} both upper- and lower-case letters, digits and preferably also symbols \emph{(\#, \_, etc.)}. Additionally, it shouldn't be a word -- or a word where an L is replaced by a 1. And of course it has to be at least 8 characters long. And you're not supposed to use the same password more than once place. With all of these rules for strong passwords, its hardly a surprise that a lot of the regular users of IT systems resort to simple, repetitive passwords.

	To help alleviate this problem, a new class of software grew popular: Password managers. Simple tools, protected by a single master password, that generates and stores passwords, in a secure manner. A lot of the IT professionals took these tools to their heart, despite their inherent flaws. 

	As with so many things, in our IT infrastructure, the users crave convenience. Tools storing an encrypted file locally, was no longer good enough, as the majority of users began to use multiple devices. Hence, the password managers slowly migrated into The Cloud.

	\section{The Cloud}
		The origin of the term ``The Cloud'' stems from Cloud Computing. Computations too heavy to be performed on a single machine, was divided into several -- usually networked -- machines, which then shared the computationally load. However, when we say ``The Cloud'' today, it is not \emph{exactly} this, that we speak of.

		The concept of ``The Cloud'' is simple: Somewhere out there, out of reach for you, in ``The Cloud'' are your data being stored, and any possible computations being done. This saves the user from the hassle of managing this, themselves. Applications such as Dropbox, OneDrive and Googe Drive is prime examples of what the could exactly is: You unload some of the ``resonsibilities'' onto something, or someone else. Once that file has been dragged into your Dropbox folder, and that little icon is green instead of blue, you're safe. Your data is now kept for you, available at all times, from any device. It is in the cloud.

		While the cloud \emph{does} come with its benefits, especially convenience, it has its own drawbacks as well. Since the user 








		In 2015 we saw what many professionals had feared: LastPass had had a leak. Thousands of user credentials had -- albeit encrypted -- been released. Panic arose, and LastPass almost forced their users to change their passwords.

		This is general issue with the cloud: You trust someone else to store your passwords. Someone else, to ensure that your data does not end up in the hand of someone else.

	\section{The Private Cloud}
		To counteract these issues, the concept of what I choose to call the Private Cloud arose. The idea is, that while a user \emph{does} what the convenience of using a cloud, he or she does not trust \emph{the} cloud. Instead, a home server is set up, providing the same features.
















	Some text.

	\section{Problem}
	\section{List of Requirements}